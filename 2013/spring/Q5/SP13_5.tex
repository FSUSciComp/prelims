\documentclass[11pt,a4paper,oneside]{article}
\usepackage{geometry}
\usepackage[pdftex]{graphicx}
\usepackage{graphicx}
\usepackage{sidecap}
\usepackage{amsmath}
\usepackage{algorithm}
\usepackage[noend]{algpseudocode}

\newenvironment{amatrix}[1]{%
  \left[\begin{array}{@{}*{#1}{c}|c@{}}
}{%
  \end{array}\right]
}

\begin{document}

\title{ISC5907 - Prelim Preparation Class \\ 
	 Spring 2013 Prelim Exam Question 5}
\date{\today}
\maketitle

\section{Question 5: Linear Algebra Prelim Question}
%##################################################
% Question 1a
\subsection{Question 1a: Deriving Equations for L and D}
\begin{align}
{\mathbf{A=LDL}^\mathrm{T}} & =
\begin{pmatrix}   1 & 0 & 0 \\
   L_{21} & 1 & 0 \\
   L_{31} & L_{32} & 1\\
\end{pmatrix}
\begin{pmatrix}   D_1 & 0 & 0 \\
   0 & D_2 & 0 \\
   0 & 0 & D_3\\
\end{pmatrix}
\begin{pmatrix}   1 & L_{21} & L_{31} \\
   0 & 1 & L_{32} \\
   0 & 0 & 1\\
\end{pmatrix} \\
& = \begin{pmatrix}   D_1 &   &(\mathrm{symmetric})   \\
   L_{21}D_1 & L_{21}^2D_1 + D_2& \\
   L_{31}D_1 & L_{31}L_{21}D_{1}+L_{32}D_2 & L_{31}^2D_1 + L_{32}^2D_2+D_3.
\end{pmatrix}
\end{align}

\begin{equation}
D_j = A_{jj} - \sum_{k=1}^{j-1} L_{jk}^2 D_k
\end{equation}
\begin{equation}
L_{ij} = \frac{1}{D_j} \left( A_{ij} - \sum_{k=1}^{j-1} L_{ik} L_{jk} D_k \right), \qquad\text{for } i>j
\end{equation}

Source: https://en.wikipedia.org/wiki/Cholesky\_decomposition

% Question 1b
\subsection{Question 1b: Pseudo-Code for $LDL^{T}$ Decomposition}
\begin{algorithm}
\caption{$LDL^{T}$ Decomposition}\label{euclid}
\begin{algorithmic}[1]
\Procedure{Compute}{}
	\State $D_1 \gets A_{11}$
	\State $L_{21} \gets \frac{A_{21}}{D_1}$
	\For{\texttt{(i in 2...n)}}:
		\State $D_i \gets A_{ii} \sum^{i-1}_{k=1} L^{2}_{ik} D_j$
		\For{\texttt{(j in i...n)}}:
			\State $L_{ij} \gets \frac{1}{D_j} (A_{ij} - \sum^{i-1}_{k=1} L_{ik} L_{jk} D_k)$
		\EndFor
	\EndFor
\EndProcedure
\end{algorithmic}
\end{algorithm}

\subsection{Question 1c: Operation Count for $LDL^{T}$ Decomposition}
This $LDL^{T}$ Decomposition requires $\frac{n^3}{3}$ operations. This makes it only about half
as expensive as the LU decomposition, which is $\frac{2n^3}{3}$ operations (see Trefethen and Bau 1997).
\hfill \break
\hfill \break \noindent
Source: https://en.wikipedia.org/wiki/Cholesky\_decomposition\#Computation \\
Wikipedia gives the number of operation counts explicitly and states, with reference, that the LDLT
decomposition uses the same number of operations.

\subsection{Question 1d: Advantages over Cholesky Decomposition}
$LDL^{T}$ Decomposition is very advantageous because it eliminates the need to compute square roots,
which can be expensive to compute and can slow down the factorization.
\hfill \break 
\hfill \break \noindent
Source: https://en.wikipedia.org/wiki/Cholesky\_decomposition\#Computation

\end{document}
