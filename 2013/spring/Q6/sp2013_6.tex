\documentclass[12pt]{article}
\usepackage{amsmath}
\usepackage{amsfonts}
\usepackage{graphicx}
\usepackage[margin=1in]{geometry}

\usepackage{listings}
\usepackage{color}

\definecolor{dkgreen}{rgb}{0,0.6,0}
\definecolor{gray}{rgb}{0.5,0.5,0.5}
\definecolor{mauve}{rgb}{0.58,0,0.82}

\lstset{frame=tb,
  language=Python,
  aboveskip=3mm,
  belowskip=3mm,
  showstringspaces=false,
  columns=flexible,
  basicstyle={\small\ttfamily},
  numbers=none,
  breaklines=true,
  breakatwhitespace=true,
  tabsize=3
}

\begin{document}
\section*{Spring 2013 Question 6}

\subsection*{Part a} 
We have to satisfy the equation $Ax=\lambda x \iff (A- \lambda I)x=0 \iff det(A- \lambda I)$.
We know that if a matrix M is triangular, then $det(M)$ is the product of the entries on the main diagonal of M.

\begin{equation}
\begin{bmatrix}
0 & a_{1,2} & a_{1,3} & \dots & a_{1,n} \\
  & \ddots & a_{2,3} & \dots & \vdots \\
  & & \ddots & \ddots & \vdots \\
  & & & \ddots & a_{n-1,n} \\
  & & & & 0
\end{bmatrix}
-
\begin{bmatrix}
\lambda & & & & \\
 & \lambda & & 0 & \\
 & & \lambda & & \\
 & 0 & & \lambda & \\
& & & & \lambda
\end{bmatrix}
=
\begin{bmatrix}
-\lambda & a_{1,2} & a_{1,3} & \dots & a_{1,n} \\
  & -\lambda & a_{2,3} & \dots & \vdots \\
  & & -\lambda & \ddots & \vdots \\
  & & & -\lambda & a_{n-1,n} \\
  & & & & -\lambda
\end{bmatrix}
\end{equation}

To satisfy $det(A- \lambda I) = (-\lambda)(-\lambda)...(-\lambda) = 0$, we must have $\lambda=0$.

\subsection*{Part b}
Again, to satisfy $Ax=\lambda x \iff Ax=0$
\begin{equation*}
x=
\begin{bmatrix}
a \\
0 \\
\vdots \\
0
\end{bmatrix}
\end{equation*}
only when A is strictly upper triangular.

\subsection*{Part c}
C can be diagonalized. $C=U \Lambda U^{-1}$
An efficient way to compute $C^100$ is to use this diagonalization such that
\begin{equation}
\begin{split}
C^2 &= U \Lambda U^{-1} U \Lambda U^{-1} \\
&= U \Lambda^2 U^{-1} \\
C^{100} &= U \Lambda^{100} U^{-1}
\end{split}
\end{equation}
Since $\Lambda$ is a diagonal matrix, each of the elements along the diagonal is simply raised to the power of the matrix, resulting in fewer computations.

\subsection*{Part d}
Defective: An $n \times n$ matrix A is a defective matrix iff it does not have $n$ linearly independent eigenvectors.

\begin{equation}
D^2 = 
\begin{bmatrix}
1 & 2 & 3 & 4 \\
0 & 1 & 2 & 3 \\
0 & 0 & 1 & 2 \\
0 & 0 & 0 & 1
\end{bmatrix}
\\
D^3 = 
\begin{bmatrix}
1 & 3 & 6 & 10 \\
0 & 1 & 3 & 6 \\
0 & 0 & 1 & 3 \\
0 & 0 & 0 & 1
\end{bmatrix}
\\
D^4 = 
\begin{bmatrix}
1 & 4 & 10 & 20 \\
0 & 1 & 4 & 10 \\
0 & 0 & 1 & 4 \\
0 & 0 & 0 & 1
\end{bmatrix}
\\
D^4 = 
\begin{bmatrix}
1 & 5 & 15 & 35 \\
0 & 1 & 5 & 15 \\
0 & 0 & 1 & 5 \\
0 & 0 & 0 & 1
\end{bmatrix}
\end{equation}
The top row of matrix $D^n$ corresponds to the nth diagonal in Pascal's triangle. If the top row of Pascal's triangle is 1, the element in the first row and last column of $D^{100}$ can be found in Pascal's triangle in the 103rd row and 4th column.
\begin{center}
\includegraphics[width=8cm]{pascalshighlighted.png}
\end{center}

\subsection*{Part e}
Let's take a 4 by 4 matrix to see if we can find a pattern using back substitution.
\begin{equation}
\begin{bmatrix}
1 & 1 & 1 & 1 \\
0 & 1 & 1 & 1 \\
0 & 0 & 1 & 1 \\
0 & 0 & 0 & 1
\end{bmatrix}
\begin{bmatrix}
x_1 \\
x_2 \\
x_3 \\
x_4
\end{bmatrix}
=
\begin{bmatrix}
b_1 \\
b_2 \\
b_3 \\
b_4
\end{bmatrix}
\end{equation}\\
First we let $x_4=b_4$. Then $x_3 + x_4 = b_3$ is equivalent to $x_3 + b_4 = b_3$ and so $x_3 = b_3 - b_4$. \\
$x_2 + x_3 + x_4 = b_2 \iff x_2 + (b_3-b_4) + b_4) = b_2 \iff x_2 + b_3 = b_2 \iff x_2 = b_2 - b_3$ \\
Similarly, $x_1 + b_2 = b_1$. \\

An algorithm would be \\
$x_n = b_n$ \\
For $i=n-1$ to $1$: \\
$\hspace{5px} x_i = b_i - b_{i+1}$

\end{document}