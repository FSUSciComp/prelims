\documentclass[a4paper]{article}

\usepackage[english]{babel}
\usepackage[utf8]{inputenc}
\usepackage{amsmath}
\usepackage{hyperref}
\usepackage{subcaption}
\usepackage{graphicx}
\usepackage[colorinlistoftodos]{todonotes}

\title{Q1. Numerical Linear Algebra: (Dr. Peterson )\\Summer 2016}


\author{Amirhessam Tahmassebi}

\date{\today}

\begin{document}
\maketitle


 

\section*{a)}


\subsection*{Stationary Iterative Method:}
$$x^{k+1} = p x^k + c$$
These are methods where the data in the equation remains fixed. Solve a linear system with an operator approximating the original one and based on a measurement of the error in the result. Jacobi, Gauss-Seidel, and Successive Over Relaxation (SOR) are three examples.
\subsubsection*{Non-Stationary Iterative Method:}
$$x^{k+1} =  x^k + \sigma_k d_k$$
These are methods where the data changes at each iteration. $\sigma_k$ and $d_k$ change for each iteration. $d_k$ called search direction and $\sigma_k$ is the step length. Krylov space method, Conjugate Gradient (CD), Generalized Minimal Residual (GMRES), BICG, and BICGSTAB are some examples of this method.


\section*{b)}
$$A = T + L +U $$
For each numerical method, we have two ways: Stable and Cheap.\\
As we recall, for Gauss-Seidel we keep a lower triangular matrix on the left hand side and move the upper triangular to the right hand side, so we could use forward substitution.\\
But, here we are dealing with a full lower triangular matrix which cost the order of $n^2$. Solving a tridiagonal matrix is the order of $n$.
$$Ax = b$$
$$(T + L + U) x = b \Rightarrow Tx = b - (L+U)x$$
$$ X ^{k+1} = -T^{-1} (L+U)x^k + T^{-1} b$$
$$\Rightarrow p =-T^{-1} (L+U)x^k , c= T^{-1} b $$


\section*{c)}

Recall SOR method:\\
Given a square system of n linear equations with unknown $x$:\\
$$Ax = b$$
A can be decomposed into a diagonal component $D$, and strictly lower and upper triangular components $L$ and $U$:
$$A = D+ L + U$$
The system of linear equations may be rewritten as:\\
$$(D+\omega L)x = \omega b - (\omega U + (\omega -1)D)x$$
$$x_i ^ {k+1} = (1-\omega ) x_i ^k + \frac{\omega}{a_{ii}}(b_i - \sum_{j<i} a_{ij}x_j^{k+1} - \sum_{j>i} a_{ij}x_j^k),with: i = 1,2,...,n$$
For finding the similarity with SOR we have: \\
$$Tx^{k+1} = b - (L + U) x^k \Rightarrow x^{k+1} - x^k = T^{-1} [b-(L+U+T)x^k]$$
We also have from SOR: $x^{k+1} = x^k + \omega(x^{k+1} - x^k)$ \\

$$ x^{k+1} = x^k +\omega (T^{-1}[b-(L+U+T)x^k])$$
$$x^{k+1} = T^{-1}[(1-\omega)T - \omega (L+U)]x^k + \omega b T^{-1}$$
It is obvious with $ \omega = 1$ we have the same formula like part b.

\section*{d)}
\begin{center}

Pseudo Code:\\
input: Matrix A, Matrix b for RHS, initial guess $x_0$\\
We have to decompose matrix A to Matrices L, U, T \\
Set the counter = 1 \\
While $Counter < P=Max_{iteration}$ \\
Calculate Right-Hand-Side: $b- (L+U)x_0 $ \\
Call TDMA Solver to solve: $Tx = RHS$ \\
Then Substitute x: $x_0 \leftarrow x$ \\
Increase the counter: $counter \leftarrow counter+1 $\\
End While \\
\end{center}
We should also know the Thomas Algorithm to solve Tridiagonal Matrix T: \\
$$ Tx =d \Rightarrow
\begin{bmatrix}
    b_{1}       & c_{1} & 0 & \dots & 0 \\
    a_{2}       & b_{2} & c_{2} & \dots & 0 \\
    0       & a_{3} & b_{3} & \dots & 0 \\
    0       & \dots  & \dots & \dots & c_{n-1} \\
    0       & 0 & \dots  & a_n & b_n \\

\end{bmatrix}
\begin{bmatrix}
    x_{1}  \\
    x_{2}    \\
    .       \\
    .       \\
    x_n \\

\end{bmatrix} =
\begin{bmatrix}
    d_{1}  \\
    d_{2}    \\
    .       \\
    .       \\
    d_n \\

\end{bmatrix}
$$
\begin{center}
TDMA Pseudo Code:\\
Forward Elimination Phase: \\
for k=2 until n do: \\
$m = \frac{a_k}{b_k -1} $ \\
$ b_k = b_k - mc_{k-1}$ \\
$ d_k = d_k -md_{k-1}$ \\
End for Loop(k) \\

Backward Substitution Phase: \\
$x_n = \frac{d_n}{b_n} $ \\

for k = n-1 stepdown until 1 do: \\
$x_k = \frac{d_k -c_k x_{k+1}}{b_k}$ \\
End Loop(k)
\end{center} 
For the Operation Count for each iteration we have: \\
1) LU:	2N multiplications,	N additions \\
2) Forward:	N multiplications,	N additions \\
3) Backward:	2N multiplications,	N additions\\
4) RHS : $N^2$ Multiplications , N Subtractions \\
Total = $9N + N^2 $ \\
For P iteration we will have: Total = $9  P N + PN^2 $ $ \sim O(N^2)$
\section*{e)}
I have tested out three different examples, Symmetric Positive Definite, Diagonally dominant matrix and a general matrix.\\
Theorem: \\
Thomas Algorithm will work if: \\
\begin{center}
$ |b_i| > |a_i| + |c_i| $  with $ i = 2 , 3, ... , N-1$ \\
$|a_1| > |c_1| and |a_N| > |c_N|$
\end{center}








\end{document}