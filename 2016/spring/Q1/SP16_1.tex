\documentclass[11pt,a4paper,oneside]{article}
\usepackage{geometry}
\usepackage[pdftex]{graphicx}
\usepackage{graphicx}
\usepackage{sidecap}
\usepackage{amsmath}
\usepackage{algorithm}
\usepackage[noend]{algpseudocode}

\newenvironment{amatrix}[1]{%
  \left[\begin{array}{@{}*{#1}{c}|c@{}}
}{%
  \end{array}\right]
}

\begin{document}

\title{ISC5907 - Prelim Preparation Class \\ 
	 Spring 2016 Prelim Exam Question 1}
\date{\today}
\maketitle

\section{Question 1: Linear Algebra LU Decomposition}
%##################################################
\subsection{Question 1a: Equations for General LU Decomposition}
\begin{align}
{\mathbf{A=LU}} & =
\begin{pmatrix}   I & 0 & 0 \\
   L_2 & I & 0 \\
   0 & L_3 & I \\
\end{pmatrix}
\begin{pmatrix}   U_1 & D_1 & 0 \\
   0 & U_2 & D_2 \\
   0 & 0 & U_3\\
\end{pmatrix} \\
& = \begin{pmatrix}
		U_1 & D_1 & 0 \\
		L_2 U_1 & L_2 D_1 + U_2 & D_2 \\
		0 & L_3 D_1 + U_2 & L_3 D_2 + U_3
\end{pmatrix} \\
& = \begin{pmatrix}
		U_1 & B_1 & 0 \\
		L_2 U_1 & L_2 B_1 + U_2 & B_2 \\
		0 & L_3 B_1 + U_2 & L_3 B_2 + U_3
\end{pmatrix} \\
& = \begin{pmatrix}
	A_1 & B_1 & 0 \\
	C_2 & A_2 & B_2 \\
	0   & C_n & A_n
\end{pmatrix}
\end{align}
	
\begin{equation}
U_j = A_j - L_j B_j
\end{equation}
\begin{equation} 
D_j = B_j
\end{equation}
\begin{equation}
L_{i} = (C_i - U_{i-1}) B_{i-2}^{-1}
\end{equation}

In this calculation, each $U_i$ and $L_i$ need not necessarily be triangular.
There is nothing in the definitions and equations above requiring them to be.

\subsection{Question 1b: Pseudo-Code for General LU Decomposition}
\begin{algorithm}[H]
\caption{Block LU Decomposition}\label{euclid}
\begin{algorithmic}[1]
\Procedure{Compute}{}
	\State $U_1 \gets A_{1}$
	\State $L_{2} \gets C_2 U_1^{-1}$
	\For{\texttt{(i in 2,...,n)}}:
		\State $U^T_{i-1} = L_\star U_\star \gets U^T_{i-1} L^T_i = C_i^T$
		\For{\texttt{(k = 1,...,p)}}:
			\State $z_k = U_\star l_k$ \texttt{\# $l_k$ is kth row-transpose of $L_i^T$}
			\State $L_\star z_k = C_k$ \texttt{\# Forward solve for $z_k$}
			\State $U_\star l_k = z_k$ \texttt{\# Back Solve}
		\EndFor
		\State $D_{i-1} \gets B_{i-1}$
		\State $U_{i-1} \gets A_{i-1} - L_i B_{i-1}$
	\EndFor
\EndProcedure
\end{algorithmic}
\end{algorithm}

\subsection{Question 1c: Algorithm for Solving $A\vec{x} = \vec{b}$}
\begin{algorithm}[H]
\caption{Block LU Solve}\label{euclid}
\begin{algorithmic}[1]
\Procedure{Compute Forward Solve}{}
	\State $A \gets L U$
	\State $y \gets L^{-1}b$ 
\EndProcedure
\Procedure{Compute Backward Solve}{}
	\State $x \gets U^{-1} y$
\EndProcedure
\end{algorithmic}
\end{algorithm}

\subsection{Question 1d: Operation Count for LU Decomposition}
  Using the algorithm above as a means to compute the operation count, we have
that the total number of operations for LU if A and B are tridiagonal and C is diagonal,
$O(2np^{3})$ and if not diagonal, $O(\frac{2}{3} np^{3})$.

\subsection{Question 1e: Example of Discretized DE System}
   An example of where you'd run into a system like this, is in discretization of
a Partial Differential Equation such as the Poisson Equation for Finite Differences
using a 5-point stencil.

\begin{equation}
	\Delta f = - \frac{\partial^2 f}{\partial x^2} - \frac{\partial^2 f}{\partial y^2} = 0
\end{equation}
\begin{equation}
	-u_{i,j-1} - u_{i-1,j} + 4u_{i,j} - u_{i,j+1} = h^2 f(i,j)
\end{equation}
\begin{equation}
	{\mathbf{A}} & =
\begin{pmatrix}   4 & -1 & \hdots & -1 & 0 & ...  \\
				  -1 & 4 & -1 & \hdots & -1 & ...  \\
				  \vdots & 0 & -1 & 4 & -1 & ...  \\
				   -1 & \ddots & 0 & \ddots & \ddots & \ddots 
\end{pmatrix}

Source: https://en.wikipedia.org/wiki/Discrete\_Poisson\_equation

\end{document}
