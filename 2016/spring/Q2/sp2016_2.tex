\documentclass[12pt]{article}
\usepackage{amsmath}
\usepackage{amsfonts}
\usepackage[margin=1in]{geometry}

\usepackage{listings}
\usepackage{color}

\definecolor{dkgreen}{rgb}{0,0.6,0}
\definecolor{gray}{rgb}{0.5,0.5,0.5}
\definecolor{mauve}{rgb}{0.58,0,0.82}

\lstset{frame=tb,
  language=Python,
  aboveskip=3mm,
  belowskip=3mm,
  showstringspaces=false,
  columns=flexible,
  basicstyle={\small\ttfamily},
  numbers=none,
  breaklines=true,
  breakatwhitespace=true,
  tabsize=3
}

\begin{document}
\section*{Spring 2016 Question 2}

\subsection*{a)} 
We are given the trapezoidal quadrature method below
\begin{equation*}
I(f) = \int_a^b f(x) dx = \dfrac{b-a}{2} \left[f(a) + f(b)\right] + K_1(b-a)^2 +
K_2(b-a)^4 + \dots
\end{equation*}
If we divide our domain $[a,b]$ using $M$ subintervals of length $h$, we can
evaluate the approximation $\hat{I}_h(f)$ for each subinterval and sum these
values over the entire domain to get $\hat{I}(f)$. The first error term is given
$\frac{f''(\xi)(b-a)}{12}(b-a)^2$ which we can use to evaluate the equispaced
trapezoid formula.
\begin{equation*}
\hat{I}(f) = \sum_{i=1}^{M} \left( \dfrac{h}{2} \left[f(a+h(i-1)) +
f(a+hi)\right] + \dfrac{f''(\xi)h^3}{12} \right)
\end{equation*}
We should get 2nd order accuracy for this method.

\subsection*{b)}
The approximation $\hat{I}(f)$ above is equivalent to $T_{M,1}$ for this
problem. To get $T_{2M,1}$ we sum over $2M$ intervals of length $h/2$.
\begin{equation*}
\begin{split}
T_{M,1} &= \sum_{i=1}^{M} \left( \dfrac{h}{2} \left[f(a+h(i-1)) + f(a+hi)\right]
+ \dfrac{f''(\xi)h^3}{12} + K_2h^4 \right) \\
&= \sum_{i=1}^{M} \dfrac{h}{2} \left[f(a+h(i-1)) + f(a+hi)\right] +
\dfrac{Mf''(\xi)h^3}{12} + MK_2h^4
\end{split}
\end{equation*}
\begin{equation*}
\begin{split}
T_{2M,1} &= \sum_{i=1}^{2M} \left( \dfrac{h}{4} \left[f(a+\frac{h(i-1)}{2}) +
f(a+\frac{hi}{2})\right] + \dfrac{f''(\xi)h^3}{96} + K_2h^4 \right) \\
&= \sum_{i=1}^{2M} \dfrac{h}{4} \left[f(a+\frac{h(i-1)}{2}) +
f(a+\frac{hi}{2})\right] + \dfrac{2Mf''(\xi)h^3}{96} + 2MK_2h^4
\end{split}
\end{equation*}
We can subtract $T_{M,1}$ from $T_{2M,1}$ in the following manner so that the
coefficients for both methods add up to 1.
\begin{equation*}
\begin{split}
T_{2M,2} &= \dfrac{4}{3} \left( T_{2M,1}-\dfrac{T_{M,1}}{4} \right) \\
&= \dfrac{4}{3} \sum_{i=1}^{2M} \dfrac{h}{4} \left[f(a+\frac{h(i-1)}{2}) +
f(a+\frac{hi}{2})\right] \\ &\hspace{1em} - \dfrac{1}{3} \sum_{i=1}^{M}
    \dfrac{h}{2} \left[f(a+h(i-1)) + f(a+hi)\right] + \dfrac{8MK_2h^4}{3} -
    \dfrac{MK_2h^4}{12}
\end{split}
\end{equation*}
We should get a 4th order accuracy for this combined method.

\subsection*{c)}
See sp2016\_2i.py and sp2016\_2ii.py for code for both parts.
The following output shows the convergence rates for part a and b approaching 2 and 4 respectively:
\begin{lstlisting}
sp2016\_2i.py:
[ 2.22026567  2.0467292   2.01125847  2.00278934]
sp2016\_2ii.py:
[ 4.3716838   4.08236705  4.02004043  4.00497702]
\end{lstlisting}

\subsection*{d)}
We can use Romberg integration to combine approximations $T_{k,i}$ where i is
the new i'th ``sequence''. In the pseudo code below, m represents the number of
step sizes.
\begin{lstlisting}
for all i in i's
  for all m in M's
    T_[m_next,i] = (4 ** (i-1) * T_[m_next] ** (i-1) - T_[m] ** (i-1)) / 4 ** (i-1) - 1
  end
end
\end{lstlisting}
For this particular problem with $i=3$, this reduces to a program with one for loop.
\begin{lstlisting}
for all m in M's
  T_[m_next] = (4 ** 2 * T_[m_next] ** 2 - T_[m] ** 2) / 4 ** 2 - 1
end
\end{lstlisting}

\end{document}
