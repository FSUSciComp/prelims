\documentclass{article}
\usepackage{hyperref}

\begin{document}

\section{Central Differnece}

First lets take the Taylor series of these two terms

\[
    f(x + h) = f(x) + f^{\prime}(x) h + 
    \frac{f^{\prime\prime}(x) h^2}{2!} +
    \frac{f^{\prime\prime\prime}(x) h^3}{3!}+
    \frac{f^{(4)}(x) h^4}{4!}+
    \frac{f^{(5)}(x) h^5}{5!}+ \dots
\]
\[
    f(x - h) = f(x) - f^{\prime}(x) h + 
    \frac{f^{\prime\prime}(x) h^2}{2!} -
    \frac{f^{\prime\prime\prime}(x) h^3}{3!}+
    \frac{f^{(4)}(x) h^4}{4!}-
    \frac{f^{(5)}(x) h^5}{5!}+ \dots
\]
and subtract the results
\[
    D_1 \equiv f(x+h) - f(x-h) = 
    2 f^{\prime}(x) h + 
    \frac{2f^{\prime\prime\prime}(x) h^3}{3!}+
    \frac{2f^{(5)}(x) h^5}{5!}+ \dots
\]
After some rearranging we find
\[
    f^{\prime}(x) = \frac{f(x+h)-f(x-h)}{2h} -
    \frac{f^{\prime\prime\prime}(x) h^2}{3!}-
    \frac{f^{(5)}(x) h^4}{5!}+ \dots
\]
\[
        f^{\prime}(x) = \frac{f(x+h)-f(x-h)}{2h} + \mathcal{O}(h^2)
\]
Following a similar procedure from above we find 
\[
    D_2 \equiv f(x+2h) - f(x-2h) = 
    4 f^{\prime}(x) h + 
    \frac{16f^{\prime\prime\prime}(x) h^3}{3!}+
    \frac{64f^{(5)}(x) h^5}{5!}+ \dots
\]
The $h$ and the $2h$ results can be compined to form
\[
    8D_1 - D_2 = f(x-2h) -8f(x-h) + 8f(x+h) - f(x+2h) = 
    12h \bigg(f^{\prime}(x) - \frac{f^{(5)}(x) h^4}{30}+ \dots \bigg)
\]
Notice that the $\mathcal{O}(h^2)$ term cancels
\[
f^{\prime} = 
    \frac{f(x-2h) -8f(x-h) + 8f(x+h) - f(x+2h)}{12h}+\mathcal{O}(h^4)
\]

\section{Numerical Results}

See python code

\section{Richardson Extrapolation}


Lets rewrite our central difference formula as such
\[
    f^{\prime}_{h}(x) = \mathcal{N}(h) -
    \frac{f^{\prime\prime\prime}(x) h^2}{6}-
    \frac{f^{(5)}(x) h^4}{120}+ \dots
\]

\[
        \mathcal{N}(h) \equiv  \frac{f(x+h)-f(x-h)}{2h}
\]
If we replace $h$ with $h/2$ we find
\[
    f^{\prime}_{h/2}(x) = \mathcal{N}(h/2) -
    \frac{f^{\prime\prime\prime}(x) h^2}{24}-
    \frac{f^{(5)}(x) h^4}{1920}+ \dots
\]
The $h$ and the $h/2$ results can be compined to form
\[
    4f^{\prime}_{h/2}(x)-f^{\prime}_{h}(x)   = 
    3f^{\prime}(x) = 4\mathcal{N}(h/2) - \mathcal{N}(h) +
    \frac{f^{(5)}(x) h^4}{160}+ \dots
\]
Notice that the $\mathcal{O}(h^2)$ term cancels just as before but the
truncation error converges much faster. Finally we are left with
\[
    f^{\prime}(x) = \mathcal{N}(h/2)+ 
    \frac{\mathcal{N}(h/2) - \mathcal{N}(h)}{3}
    + \mathcal{O}(h^4)
\]
For further reading I suggest
\url{https://www.math.washington.edu/~greenbau/Math_498/lecture04_richardson.pdf}

\section{Lanczos Formula}
\[
    g^{\prime}(x) = \lim_{h \rightarrow 0} \frac{3}{2h^3}
    \int_{-h}^{h} t g(x+t) dt
\]
Newton-Cotes integration rule, 
\[
    \int_{-h}^{h} f(x)dx = 
    \frac{h}{4} \left(
            f(-h) + 3f\left(-\frac{h}{3}\right) + 3f\left(\frac{h}{3}\right) + f(h)
\right)
    + \mathcal{O}(h^5)
\]
Using NC and the definition of $g^{\prime}$ show 
\[
        g^{\prime}(x) =  \frac{A}{4} \left(
            -g(x-h) 
            -g\left(x-\frac{h}{3}\right) +
            g\left(x+\frac{h}{3}\right) +
            g(x+h)
\right)
    + \mathcal{O}(h^p)
\]
find $A$ and $p$

Lets plug the NC formula into our definition of $g^{\prime}$
\[
        g^{\prime}(x) =  \frac{3}{2h^3}\left[
    \frac{h}{4} \left(
            -hg(x-h) + 
            3\left(-\frac{h}{3}\right)g\left(x-\frac{h}{3}\right) +
            3\left(\frac{h}{3}\right)g\left(x+\frac{h}{3}\right) +
            hg(x+h)
\right)
    + \mathcal{O}(h^5)
    \right]
\]
and pull out a $h$
\[
        g^{\prime}(x) =  \frac{3}{2h^3}\left[
    \frac{h^2}{4} \left(
            -g(x-h)
            -g\left(x-\frac{h}{3}\right) +
            g\left(x+\frac{h}{3}\right) +
            g(x+h)
\right)
    + \mathcal{O}(h^5)
    \right]
\]
We are then left with 
\[
        g^{\prime}(x) =  \frac{3}{2h}\frac{1}{4} \left(
            -g(x-h) 
            -g\left(x-\frac{h}{3}\right) +
            g\left(x+\frac{h}{3}\right) +
            g(x+h)
\right)
    + \mathcal{O}(h^2)
\]
\[
        A =\frac{3}{2h} 
\]
\[
        p = 2
\]
\end{document}
