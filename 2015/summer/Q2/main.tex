\documentclass[a4paper]{article}

\usepackage[english]{babel}
\usepackage[utf8]{inputenc}
\usepackage{amsmath}
\usepackage{hyperref}
\usepackage{graphicx}
\usepackage[colorinlistoftodos]{todonotes}

\title{Q2. Linear Algebra: (Dr. Wang )\\Summer 2015}


\author{Amirhessam Tahmassebi}

\date{\today}

\begin{document}
\maketitle

Consider the following matrix:

\[A=
\begin{bmatrix}
  1 & 0 & 0 \\
  1 & 0 & 1 \\
  0 & 1 & 0 \\
\end{bmatrix}
\]

Part-a): \\
Using mathematical induction we have: \\

$$n=3 \longrightarrow A^3 = A + A^2 -I$$
$$p(3) = A^3 = A + A^2 -I$$ our formula is true for n=3
$$p(k) = A^k = A^{k-2} + A^2 - I$$ we assume that is true for n = k
$$p(k+1) = A^{k+1} = A^{k-1} +A^2 -I$$ we are going to show that it is true for n = k+1. \\

$$Ap(k) = A(A^k) = A(A^{k-2} + A^2 -I) = A^{k-1} + A^3 - A$$ Now we use our $p(3)$ here:

$$\Longrightarrow Ap(k) = A^{k-1} + A + A^2 - I - A =A^{k-1} + A^2 - I =p(k+1)$$




Part-b): \\

For calculating $A^{100}$ we have:\\

Using formula: $ A^n = A^{n-2} + A^2 -I$ we have: \\
$$n=4 \longrightarrow A^4 = A^2+A^2 - I = 2A^2-I$$
$$n=6 \longrightarrow A^6 = A^4+A^2 - I = 2A^2-I + A^2 - I = 3A^2 - 2I$$
$$n=8 \longrightarrow A^8 = A^6+A^2 - I =3A^2 - 2I + A^2 -I = 4A^2 -3I$$
$$\vdots$$
$$\vdots$$
$$A^{N} = \frac{N}{2}A^2 - (\frac{N}{2} -1)I$$

Here for$ N = 100$ we have:

$$A^{100} = 50A^2 - 49I$$

We have:\\


\[A^2=
\begin{bmatrix}
  1 & 0 & 0 \\
  1 & 1 & 0 \\
  1 & 0 & 1 \\
\end{bmatrix}
\]


So, we have to just multiply $A^2$ with 50 and subtract from $49I$, then we will have: \\


\[A^{100}=
\begin{bmatrix}
  1 & 0 & 0 \\
  50 & 1 & 0 \\
  50 & 0 & 1 \\
\end{bmatrix}
\]


















\end{document}