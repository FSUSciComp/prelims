\documentclass[12pt]{article}
\usepackage{amsmath}
\usepackage{amsfonts}
\usepackage[margin=1in]{geometry}

\usepackage{listings}
\usepackage{color}

\definecolor{dkgreen}{rgb}{0,0.6,0}
\definecolor{gray}{rgb}{0.5,0.5,0.5}
\definecolor{mauve}{rgb}{0.58,0,0.82}

\lstset{frame=tb,
  language=Python,
  aboveskip=3mm,
  belowskip=3mm,
  showstringspaces=false,
  columns=flexible,
  basicstyle={\small\ttfamily},
  numbers=none,
  breaklines=true,
  breakatwhitespace=true,
  tabsize=3
}

\begin{document}
\section*{Spring 2015 Question 2}

\subsection*{Part a} 
We have to show that A is positive definite by showing $\vec{u}^TA\vec{u} > 0$.
\begin{equation}
\vec{u}^TA = 
\begin{bmatrix}
u_1 & u_2 & \dots & u_{m-1} & u_m
\end{bmatrix}
\begin{bmatrix}
\frac{2}{h^2} & \frac{-1}{h^2} & 0 \\
\frac{-1}{h^2} & \frac{2}{h^2} & \ddots \\
0 & \ddots & \ddots 
\end{bmatrix}
\end{equation}
\begin{equation}
\vec{u}^TA = 
\dfrac{1}{h^2} 
\begin{bmatrix}
2u_1-u_2 & \left[-u_{i-1} + 2u_i - u_{i+1}\right]_{i=2}^{m-1} & 2u_m-u_{m-1}
\end{bmatrix}
\end{equation}
\begin{equation}
\vec{u}^TA\vec{u} = 
\dfrac{1}{h^2} 
\begin{bmatrix}
2u_1-u_2 & \left[-u_{i-1} + 2u_i - u_{i+1}\right]_{i=2}^{m-1} & 2u_m-u_{m-1}
\end{bmatrix}
\begin{bmatrix}
u_1 \\ u_2 \\ \vdots \\ u_{m-1} \\ u_m
\end{bmatrix}
\end{equation}
\begin{equation}
\begin{split}
\vec{u}^TA\vec{u} &= 
\dfrac{1}{h^2} 
\left( 2u_1^2-u_1u_2 + \sum_{i=2}^{m-1} \left[-u_{i-1}u_i + 2u_i^2 - u_iu_{i+1}\right] + 2u_m^2-u_{m-1}u_m \right) \\
&= 
\dfrac{1}{h^2} 
\left( u_1^2 + u_1^2 - u_1u_2 + \sum_{i=2}^{m-1} \left[-u_{i-1}u_i + u_i^2 + u_i^2 - u_iu_{i+1}\right] + u_m^2 + u_m^2 -u_{m-1}u_m \right) \\
&= 
\dfrac{1}{h^2} 
\left( u_1^2 + \sum_{i=1}^{m-1} \left[ u_i^2 - u_iu_{i-1} - u_{i-1}u_i + u_{i+1}^2 \right] + u_m^2 \right) \\
&= 
\dfrac{1}{h^2} 
\left( u_1^2 + \sum_{i=2}^{m-1} \left( u_i - u_{i-1} \right)^2 + u_m^2 \right)
\end{split}
\end{equation}
All the squared terms are positive so $\vec{u}^TA\vec{u}$ is positive.

\subsection*{Part b}
\subsubsection*{i}
Assume the iterative method stated in the problem doesn't converge such that 
\begin{equation}
\lim_{k \rightarrow \infty} ||\vec{x}^k - \vec{x}|| = \lim_{k \rightarrow \infty} e^k \neq 0
\end{equation}
Since $x^{k+1} = Px^k + c$,
\begin{equation}
\begin{split}
e^{k} &= ||\vec{x}^k - \vec{x}|| \\
&= || \vec{x}^{k-1} + c - (P\vec{x} + c) || \\
&= ||P(\vec{x}^{k-1} - \vec{x})|| \\ 
&= ||Pe^{k-1}|| \\
&= ||P^2e^{k-2}|| \\
& \hspace{0.5em} \vdots \\
&= ||P^ke^0|| \\
&= ||P^k|| \\
\end{split}
\end{equation}
Since we assumed $\lim_{k \rightarrow \infty} e^k \neq 0$, $\lim_{k \rightarrow \infty} ||P^k|| \neq 0$. But we are given that $\lim_{k \rightarrow \infty} ||P^k||_\alpha = 0_{n \times n}$ in the question. So we have a contradiction. Therefore the method converges.
\subsubsection*{ii}
We have the general problem $B\vec{x} = \vec{f}$. Using the splitting method $B=M-N$, we show
\begin{equation}
\begin{split}
B\vec{x} &= \vec{f} \\
M\vec{x} - N\vec{x} &= \vec{f} \\
\vec{x} &= M^{-1}\vec{f} + M^{-1}N\vec{x}
\end{split}
\end{equation}
which is the Jacobi method $\vec{x}^{k+1} =  M^{-1}N\vec{x}^k + M^{-1}\vec{f}$ with M being a diagonal matrix and $N$ being all the off-diagonal elements (lower and upper). When we substitute $P = M^{-1}N$ and $c = M^{-1}\vec{f}$ we get the iterative method from our problem $\vec{x}^{k+1} = P\vec{x}^k + \vec{c}$. For our tridiagonal matrix $A$ we have $P_A = 0_{n \times n}$.
\begin{equation}
P_A = 
\begin{bmatrix}
\frac{h^2}{2} & 0 & 0\\
0 & \frac{h^2}{2} & \ddots \\
0 & \ddots & \ddots
\end{bmatrix}
\begin{bmatrix}
0 & \frac{-1}{h^2} & 0\\
\frac{-1}{h^2} & 0 & \ddots \\
0 & \ddots & \ddots
\end{bmatrix}
= 
\begin{bmatrix}
0 & 0 & \hdots \\
0 & 0 & \ddots \\
\vdots & \ddots & \ddots
\end{bmatrix}
\end{equation}

\subsection*{Part c}
\subsubsection*{i}
Our matrix $A$ (in part a) is symmetric positive definite so the splitting $A = M - N$ has $M$ containing the diagonal elements of $A$, and $-N$ containing all the subdiagonal elements of $A$. So $M=M^T$ which means $M^T-N$ is also s.p.d. (as $M-N$ is also s.p.d. which can be shown by part a above). By the Lemma given, the spectral radius $\rho(M^{-1}N) < 1$. This iterative method converges iff for matrix $P$, $\rho(P)<1$. In part b.ii we showed $P_A$ is a matrix of 0s so $\rho(P) = 0$.
\subsubsection*{ii}
To find which matrix will converge faster we can find the spectral radius of each matrix. The matrix with the smaller spectral radius converges faster; in this case, $\rho(B_1)=6$ and $\rho(B_1)=10$, so $B_1$ will converge faster using the Jacobi method.
\subsection*{Part d}
We can use the power method to find the spectral radius. Using the file \begin{verbatim}sp2015_2d.py \end{verbatim} we get the number of iterations for convergence of the power method, the dominant eigenvector, and the corresponding eigenvalue.
\begin{lstlisting}
After 8178 iterations.
--------------------------
Dominant eigenvector:
[[-0.03077979]
 [ 0.06153076]
 [-0.09222411]
 [ 0.12283109]
 [-0.15332301]
 [ 0.18367131]
 [-0.21384751]
 [ 0.2438233 ]
 [-0.27357053]
 [ 0.30306126]
 [-0.33226774]
 [ 0.36116247]
 [-0.38971823]
 [ 0.41790806]
 [-0.44570533]
 [ 0.47308375]
 [-0.50001737]
 [ 0.52648063]
 [-0.55244836]
 [ 0.57789585]
 [-0.60279879]
 [ 0.62713339]
 [-0.65087631]
 [ 0.67400475]
 [-0.69649644]
 [ 0.71832967]
 [-0.73948329]
 [ 0.75993677]
 [-0.77967019]
 [ 0.79866427]
 [-0.81690037]
 [ 0.83436056]
 [-0.85102758]
 [ 0.86688489]
 [-0.88191668]
 [ 0.89610789]
 [-0.90944422]
 [ 0.92191215]
 [-0.93349896]
 [ 0.94419275]
 [-0.95398243]
 [ 0.96285774]
 [-0.9708093 ]
 [ 0.97782857]
 [-0.98390791]
 [ 0.98904052]
 [-0.99322055]
 [ 0.99644303]
 [-0.99870389]
 [ 1.        ]
 [-1.00032916]
 [ 0.99969009]
 [-0.99808245]
 [ 0.99550686]
 [-0.99196487]
 [ 0.98745898]
 [-0.98199263]
 [ 0.97557024]
 [-0.96819713]
 [ 0.95987962]
 [-0.95062493]
 [ 0.94044124]
 [-0.92933767]
 [ 0.91732425]
 [-0.90441195]
 [ 0.89061265]
 [-0.87593912]
 [ 0.86040504]
 [-0.84402496]
 [ 0.82681433]
 [-0.80878943]
 [ 0.78996739]
 [-0.77036616]
 [ 0.75000453]
 [-0.72890204]
 [ 0.70707906]
 [-0.68455666]
 [ 0.66135669]
 [-0.6375017 ]
 [ 0.61301492]
 [-0.58792026]
 [ 0.56224228]
 [-0.53600616]
 [ 0.50923767]
 [-0.48196314]
 [ 0.45420946]
 [-0.42600402]
 [ 0.39737471]
 [-0.36834985]
 [ 0.3389582 ]
 [-0.30922893]
 [ 0.27919155]
 [-0.24887591]
 [ 0.21831216]
 [-0.18753074]
 [ 0.15656229]
 [-0.12543768]
 [ 0.09418793]
 [-0.06284421]
 [ 0.03143779]]
Spectral radius:
[[ 40794.13036297]]
\end{lstlisting}

\end{document}