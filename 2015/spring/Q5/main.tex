\documentclass[a4paper]{article}

\usepackage[english]{babel}
\usepackage[utf8]{inputenc}
\usepackage{amsmath}
\usepackage{hyperref}
\usepackage{graphicx}
\usepackage[colorinlistoftodos]{todonotes}

\title{Q5. Numerical Quadrature: (Dr. Shanbhag )}

\author{Amirhessam Tahmassebi}

\date{\today}

\begin{document}
\maketitle



\section{Method}

For this problem, I have used Green's Theorem:\\
Let C be a positively oriented, piece-wise smooth, simple closed curve in a plane, and let D be the region bounded by C. If L and M are functions of (x, y) defined on an open region containing D and have continuous partial derivatives there, then:\\

$$\oint_C (L\,dx + M\,dy) = \iint_S (\frac{\partial M}{\partial x}-\frac{\partial L}{\partial y}) \,dx\,dy$$\\
where the path of integration along C is counterclockwise.\\

So, easily we can implement our integrals for area, central positions and gyration radius. \\\\
Area:
$$Area = \iint_S  \,dx\,dy$$
$$\Longrightarrow (\frac{\partial M}{\partial x}-\frac{\partial L}{\partial y}) = 1 \Longrightarrow (M=x \quad \textrm{and} \quad L=0 )$$ 
$$\Longrightarrow Area = \oint_C  x\,dy $$\\

$X_{cm}$:
$$X_{cm} =\frac{1}{Area}\iint_S x \,dx\,dy$$
$$\Longrightarrow (\frac{\partial M}{\partial x}-\frac{\partial L}{\partial y}) = x \Longrightarrow (M=\frac{x^2}{2} \quad \textrm{and} \quad L=0 )$$ 
$$\Longrightarrow X_{cm} = \frac{1}{Area}\oint_C  \frac{x^2}{2}\,dy $$\\


$Y_{cm}$:
$$Y_{cm} =\frac{1}{Area}\iint_S y \,dx\,dy$$
$$\Longrightarrow (\frac{\partial M}{\partial x}-\frac{\partial L}{\partial y}) = y \Longrightarrow (M=0 \quad \textrm{and} \quad L=-\frac{y^2}{2} )$$ 
$$\Longrightarrow Y_{cm} = \frac{-1}{Area}\oint_C  \frac{y^2}{2}\,dy $$
But, here we need to be careful about path C which is always counter clockwise. At the end, we will come up with another negative sign due to the path.
$$\Longrightarrow Y_{cm} = \frac{1}{Area}\oint_C  \frac{y^2}{2}\,dy $$

$X_{g}^2$:
$$X_{g}^2 =\frac{1}{Area}\iint_S (x-X_{cm})^2 \,dx\,dy$$
$$\Longrightarrow (\frac{\partial M}{\partial x}-\frac{\partial L}{\partial y}) = (x-X_{cm})^2 \Longrightarrow (M=\frac{(x-X_{cm})^3}{3} \quad \textrm{and} \quad L=0 )$$ 
$$\textrm{Due to the path}\Longrightarrow X_{g}^2 = \frac{-1}{Area}\oint_C  \frac{(x-X_{cm})^3}{3}\,dy $$\\

$Y_{g}^2$:
$$Y_{g}^2 =\frac{1}{Area}\iint_S (y-Y_{cm})^2 \,dx\,dy$$
$$\Longrightarrow (\frac{\partial M}{\partial x}-\frac{\partial L}{\partial y}) = (y-Y_{cm})^2 \Longrightarrow (M=0 \quad \textrm{and} \quad L=-\frac{(y-Y_{cm})^3}{3} )$$ 
$$\textrm{Due to the path }\Longrightarrow Y_{g}^2 = \frac{1}{Area}\oint_C  \frac{(y-Y_{cm})^3}{3}\,dy $$\\

$$R_{Gyration}^2 = X_{g}^2+Y_{g}^2$$



\section{Results}

The Area = 565.486677646 \\
The $X_{cm}$ = 2.63488313796e-16 \\
The $Y_{cm}$ = 0.833333333333 \\
The $R_{G}$ = 101.433333333 \\






\end{document}